%%%%%%%%%%%%%%%%%%%%%%%%%%%%%%%%%%%%%%%%%%%%%%%%%%%%%%%%%%%%%%%%%%%%%%%%%%%%%%%%
\documentclass[presentation]{beamer} %\mode<presentation>{\usetheme{sapere}}
\usetheme{CambridgeUS}
\usecolortheme{orchid}

\definecolor{themeColor}{HTML}{295D98}

\setbeamercolor*{structure}{bg=black,fg=themeColor}

\setbeamercolor*{palette primary}{use=structure,fg=white,bg=structure.fg}
\setbeamercolor*{palette secondary}{use=structure,fg=white,bg=structure.fg!75}
\setbeamercolor*{palette tertiary}{use=structure,fg=white,bg=structure.fg!50!black}
\setbeamercolor*{palette quaternary}{fg=white,bg=black}

\setbeamercolor{section in toc}{fg=black,bg=white}
\setbeamercolor{alerted text}{use=structure,fg=structure.fg!50!black!80!black}

\setbeamercolor{titlelike}{parent=palette primary,fg=structure.fg!50!black}
\setbeamercolor{frametitle}{bg=structure.fg!10!white,fg=structure.fg!50!black!80!black}

\setbeamercolor*{titlelike}{parent=palette primary}


\usepackage[utf8]{inputenc}
\usepackage{amssymb}
\usepackage{graphicx}
\usepackage{subfigure}
\usepackage{multirow}
\usepackage{hhline}
\usepackage{amsfonts,amstext,amssymb,wasysym}
\usepackage{fancyvrb}
\usepackage{alltt}
\usepackage{textcomp}
\usepackage{url}
\usepackage{multimedia,pgf}
\usepackage{geometry}
\usepackage{listings}
\usepackage{framed}
\usepackage{cleveref}

\title[Integrating DES and MABS]{Extending the Gillespie's Stochastic Simulation Algorithm for Integrating Discrete-Event and Multi-Agent Based Simulation}

\author[Montagna, Omicini, Pianini]{
Sara Montagna, Andrea Omicini, \textbf{Danilo Pianini}
\\
\texttt{{\footnotesize [sara.montagna, andrea.omicini, danilo.pianini]@unibo.it}}}


\institute[UNIBO]
{\textsc{Alma Mater Studiorum}---Universit\`a di Bologna a Cesena}

\date[2015-05-15 MABS]{The XVI International Workshop on Multi-Agent Based Simulation\\
\scriptsize May 5, 2015 - Istanbul, Turkey
}

\pgfdeclareimage[height=0.625cm]{university-logo}{imgs/logo}
\logo{\pgfuseimage{university-logo}}


\begin{document}


%===============================================================================
\frame[label=coverpage]{\titlepage}
%===============================================================================

\section*{Outline}
%===============================================================================
\frame{\tableofcontents}

\section{Introduction}
\begin{frame}{Two intuitions}
  \begin{block}{Unique conceptial framework}
    Event-driven systems and multi-agent systems are amenable of a coherent interpretation within a unique conceptual framework
  \end{block}
  \begin{block}{Powerful simulation framework}
    From the integration of Discrete Event Simulation (DES) and Multi-Agent Based Simulation (MABS)
  \end{block}
\end{frame}

\begin{frame}{Motivation: why event driven?}
  \begin{block}{Efficiency}
    \begin{itemize}
      \item Time passes fixed time steps, even if no action changes the state happen in between
      \item Modellers must carefully choose temporal granularity
      \item If there is a wide spectrum of time scales, a low granularity may ruin results, while a high granularity may lead to a waste of computational resources
    \end{itemize}
  \end{block}
\end{frame}

\begin{frame}{Motivation: why event driven?}
  \begin{block}{Accuracy, validity, coherency}
    \begin{itemize}
      \item To be as close as possible to the MAS paradigm, actions and interactions should be conducted concurrently
      \item In a time-driven setup, all the events happening in the same $\Delta{}t$, are executed (along with the environment evolution) together, possibly losing ordering and changing the system outcome
      \item Event driven patches the problem, limiting it to those actions that happen at the exact same time.
    \end{itemize}
  \end{block}
  \begin{block}{Congruence}
    \begin{itemize}
      \item Updating all the entities of the system simultaneously is often an approximation too far from reality
    \end{itemize}
  \end{block}
\end{frame}

\section{Unified computational model}

\begin{frame}{Bulding on SSAs}
  \begin{block}{Gillespie's algorith}
    \begin{itemize}
      \item Gillespie \cite{gillespie1977} first proposed an event driven stochastic simulation algorithm (SSA) for the exact stochastic simulation of chemical systems
      \item Gibson and Bruck \cite{gibson2000} improved its performance
      \begin{itemize}
	\item Next reaction selection not by propensity (function of concentration of reagents and a markovian rate) but by generated putative times
	\item Dependency graph meant to update only the events whose scheduling time might have changed because of other events
      \end{itemize}
      \item Building on their work, we extended the algorithm in order to be able to shift from the wolrd of chemistry to the richer MABS world
    \end{itemize}
  \end{block}
\end{frame}

\subsection{Model}
\begin{frame}{Generalised chemistry}
\begin{block}{Pure chemistry vs. agent-based systems}
\begin{itemize}
 \item Single, static compartment versus multiple, possibly mobile, and interconnected agents whose ability to communicate may depend on environmental and technological factors
 \item Molecules are described by concentrations (an integer), agents may carry and process any kind of data
 \item Reactions ``scheduling'' in nature follows a Poisson distribution whose rate equation depends on reagents' concentration \cite{gillespie1977}. Events in an agent-based simulation may be influenced by any of the environment components and follow any probability distribution (triggers, timers, events with memory)
 \item Agents live in an environment, such abstraction is absent in chemistry
\end{itemize}
\end{block}
Yes, it is a nicely big leap
\end{frame}

\begin{frame}{Close the gap: environment}
  \begin{figure}
    \includegraphics[
%       width=\textwidth,
      height=0.8\textheight
      ]{imgs/model} 
  \end{figure}
\end{frame}

\begin{frame}{Close the gap: reactions}
  \begin{figure}
    \includegraphics[
      width=\textwidth,
%      height=0.8\textheight
      ]{imgs/reaction} 
  \end{figure}
\end{frame}

\begin{frame}{Flexibility and data types}
  \begin{block}{Abstract Concentration}
    \begin{itemize}
    \item Concentration can be any data type
      \begin{itemize}
        \item Pick integers, the result is a simulator for (bio)chemistry, with multiple intercommunicating compartments situated in an environment \cite{drosophila}
        \item Pick ``set of tuples matching a tuple template'', the result is a simulator of network of programmable tuple spaces
        \item Pick ``any object'', the result is flexibile enough to simulate a network of devices running their own program \cite{protelis}
      \end{itemize}
    \item For each type of concentration, a specific set of legal conditions and actions can (must) be defined
    \item All the other entities can be defined in a generic fashion, and reused 
    \end{itemize}
  \end{block}
\end{frame}

\subsection{Engine}

\begin{frame}{Extended SSA phase 1: pick the next event}
  \begin{block}{How to select the next event?}
    \begin{itemize}
      \item Most high-performance SSAs presume an underlying model that only includes memoryless events \cite{slepoy2008}
      \item Gibson/Bruck's ``next reaction'' uses putative times instead
      \item We extended it adding support for addition and removal of events at runtime
      \begin{itemize}
        \item Agents may join and leave the system at runtime, new agents may be equipped with novel behaviours
      \end{itemize}
    \end{itemize}
  \end{block}
\end{frame}

\begin{frame}{Extended SSA phase 2: dependency management}
  \begin{block}{How to select the next event?}
    \begin{itemize}
      \item The dependency graph is key for the high performance of SSAs \cite{slepoy2008}
      \item In general, it is very hard to build a dependency graph in an open environment composed of multiple entities
      \item We extended the original concept of (static) dependency graph with:
      \begin{itemize}
        \item Events can be added and removed at runtime, the graph is dynamically updated
        \item Execution contexts: \texttt{local}, \texttt{neighborhood}, \texttt{global}
        \item Separation of influencing context and context of influence (input and output)
        \item Overall, the dependency graph is greatly pruned, with positive impact on performance
      \end{itemize}
    \end{itemize}
  \end{block}
\end{frame}


\section{Case study}
\subsection{A possible implementation: the Alchemist}

\begin{frame}{Alchemist}
  \begin{block}{Chemical-inspired meta simulator}
    \begin{itemize}
      \item Based on the machinery already described
      \item Java written
      \item Provides out of the box support for simulating distributed programmable tuple spaces and Protelis \cite{protelis} devices
      \item Supports mobility and complex environments, both indoor and outdoor (with data from OpenStreetMap)
      \item Available as Maven artifact (\texttt{it.unibo.alchemist:alchemist})
    \end{itemize}
  \end{block}
\end{frame}

\begin{frame}{Architecture}
  \begin{figure}
    \includegraphics[width=\textwidth]{imgs/architecture} 
  \end{figure}
\end{frame}

\subsection{Example scenario: context sensitive crowd steering in a urban environment}

\begin{frame}{Crowd-sensitive user steering}
{\footnotesize Steering against GPS traces taken at Vienna City Marathon 2013}
\begin{center}
  \movie[width=12cm,height=6.6cm,showcontrols=true,poster]{}{imgs/30.webm}
\end{center}
\end{frame}

\section{Conclusion}
\begin{frame}{Conclusion}
  \begin{block}{Integration of DES and MABS}
    \begin{itemize}
      \item We adopted an extension of Gillespie's SSA as stochastic event-driven algorithm
      \item We extended it to support the inherent complexity of multiagent systems, still retaining the performance optimisations
      \item We extended the chemical model towards higher flexibility, introducing the enviromemnt, generalising the concept of reaction and allowing arbitrary data to be a ``concentration''
      \item We implemented those concepts inside the Alchemist framework
      \item A non-trivial example was provided: crowd steering in London
    \end{itemize}
  \end{block}
\end{frame}



\section*{\refname}
%===============================================================================
\begin{frame}[allowframebreaks]
%\begin{frame}[t,allowframebreaks]
  \frametitle{\refname}
  \scriptsize
  \bibliographystyle{alpha}
  \bibliography{bibliography}
\end{frame}
\section*{\refname}






\end{document}