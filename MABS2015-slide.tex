%%%%%%%%%%%%%%%%%%%%%%%%%%%%%%%%%%%%%%%%%%%%%%%%%%%%%%%%%%%%%%%%%%%%%%%%%%%%%%%%
\documentclass[presentation]{beamer} %\mode<presentation>{\usetheme{sapere}}
\usetheme{CambridgeUS}
\usecolortheme{orchid}

\definecolor{themeColor}{HTML}{295D98}

\setbeamercolor*{structure}{bg=black,fg=themeColor}

\setbeamercolor*{palette primary}{use=structure,fg=white,bg=structure.fg}
\setbeamercolor*{palette secondary}{use=structure,fg=white,bg=structure.fg!75}
\setbeamercolor*{palette tertiary}{use=structure,fg=white,bg=structure.fg!50!black}
\setbeamercolor*{palette quaternary}{fg=white,bg=black}

\setbeamercolor{section in toc}{fg=black,bg=white}
\setbeamercolor{alerted text}{use=structure,fg=structure.fg!50!black!80!black}

\setbeamercolor{titlelike}{parent=palette primary,fg=structure.fg!50!black}
\setbeamercolor{frametitle}{bg=structure.fg!10!white,fg=structure.fg!50!black!80!black}

\setbeamercolor*{titlelike}{parent=palette primary}


\usepackage[utf8]{inputenc}
\usepackage{amssymb}
\usepackage{graphicx}
\usepackage{subfigure}
\usepackage{multirow}
\usepackage{hhline}
\usepackage{amsfonts,amstext,amssymb,wasysym}
\usepackage{fancyvrb}
\usepackage{alltt}
\usepackage{textcomp}
\usepackage{url}
\usepackage{multimedia,pgf}
\usepackage{geometry}
\usepackage{listings}
\usepackage{framed}
\usepackage{cleveref}

\definecolor{Fuchsia}{HTML}{8C368C}
\definecolor{OliveGreen}{HTML}{3C8031}

% Code highlighting
\newcommand{\il}[1]{{\it \textcolor{gray}{// #1}}} % inline comment
\newcommand{\km}[1]{\textcolor{purple}{#1}} % key mechanism primitives
\newcommand{\ex}[1]{\textcolor{blue}{#1}} % external imported Java values
\newcommand{\fc}[1]{\textcolor{Fuchsia}{#1}} % field calculus calls
\newcommand{\fn}[1]{\textcolor{blue}{#1}} % building block / function calls
\newcommand{\vb}[1]{\textcolor{OliveGreen}{#1}} % variables
\newcommand{\str}[1]{\textcolor{darkgray}{#1}} % strings

\newcommand{\bral}{\textrm{{\tt {\char '173}}}\,}
\newcommand{\brar}{\textrm{{\tt {\char '175}}}}
\newcommand{\var}{\texttt{x}}
\newcommand{\asgK}{~\texttt{=}~}
\newcommand{\letK}{\texttt{let}~}
\newcommand{\tupK}[1]{\texttt{[}#1\texttt{]}}
\newcommand{\lambdaK}[2]{\texttt{(}#1\texttt{)->}#2}
\newcommand{\bodyK}[1]{\bral\! #1\!\brar}
\newcommand{\dotK}{\texttt{.}}
\newcommand{\applyK}{\texttt{apply}}
\newcommand{\mname}{\ex{\texttt{m}}}
\newcommand{\aname}{\ex{\texttt{\#a}}}
\newcommand{\repK}[3]{\texttt{\fc{rep}(#1<-#2)}#3}
\newcommand{\ifK}[3]{\texttt{\fc{if}}(#1)#2\,\texttt{\fc{else}}\,#3}
\newcommand{\muxK}[3]{\texttt{\fc{mux}}(#1)#2\,\texttt{\fc{else}}\,#3}
\newcommand{\nbrK}[1]{\texttt{\fc{nbr}}#1}

\title[Gillespie's SSA to Integrate DES and MABS]{Extending the Gillespie's Stochastic Simulation Algorithm for Integrating Discrete-Event and Multi-Agent Based Simulation}

\author[Montagna, Omicini, Pianini]{
Sara Montagna, Andrea Omicini, \textbf{Danilo Pianini}
\\
\texttt{{\footnotesize sara.montagna@unibo.it}}}


\institute[UNIBO]
{\textsc{Alma Mater Studiorum}---Universit\`a di Bologna a Cesena}

\date[2015-05-15 MABS]{The XVI International Workshop on Multi-Agent Based Simulation\\
\scriptsize May 5, 2015 - Istanbul, Turkey
}

\pgfdeclareimage[height=0.625cm]{university-logo}{imgs/logo}
\logo{\pgfuseimage{university-logo}}


\begin{document}


%===============================================================================
\frame[label=coverpage]{\titlepage}
%===============================================================================

\section*{Outline}
%===============================================================================
\frame{\tableofcontents}



%===============================================================================
\section{Background}
%===============================================================================

%-------------------------------------------------------------------------------
\subsection{Discrete Event Simulation (DES)}
%-------------------------------------------------------------------------------


\begin{frame}{DES}

\ldots The most used approach in the simulation mainstream
\begin{block}{}
	\begin{itemize}
    		\item Instantaneous events responsible for the changes in the system state
    		\item In between events, no change to the system is assumed to occur
    		\item Different events cannot be simultaneous
		\item It is normally very efficient since it allows to jump in time from one relevant event
	\end{itemize}
\end{block}

		
\emph{Defining a DES means to model the behaviour of a system as an ordered sequence of non-continuous events, by specifying for each of them the \emph{perturbations} in the system state it provokes, and the exact point in time when it has to be triggered. }

\end{frame}


%-------------------------------------------------------------------------------
\subsection{Agent Based Modelling (ABM) and Simulation (MABS)}
%-------------------------------------------------------------------------------

\begin{frame}{MABS}

\ldots Introduced as a novel and alternative approach to Discrete Event Simulation (DES) and to Systems Dynamic (SD)

\begin{block}{}
	\begin{itemize}
		\item More than a simulation method...it is a \textbf{philosophy}...    
    		\item ABM models the system at the \emph{micro-level}
		\begin{itemize}
			\item active entity are autonomous and interacting agent, 
			\item dynamic environment
			\item global system-level behaviour as a result of the agent-to-agent interactions
		\end{itemize}
	\end{itemize}
\end{block}

\begin{block}{The simulation platform}
	\begin{itemize}
		\item Various platforms were developed for the purpose \cite{simulationtoolkits-survey,Railsback:simulation2006}, 
		\begin{itemize}
			\item  tools for developing, editing, and executing ABM, as well as for visualising the simulation dynamic. 
		\end{itemize}
		\item How do they operate over a timeline?
		\item How are agents and environment behaviours coupled and scheduled?
	\end{itemize}
\end{block}


\end{frame}


%===============================================================================
\section{Motivation}
%===============================================================================


\begin{frame}{}
\end{frame}

%===============================================================================
\section{A Unified Stochastic Computational Model}
%===============================================================================


\begin{frame}{}
\end{frame}


%-------------------------------------------------------------------------------
\subsection{Gillespie's SSA as an event-driven algorithm}
%-------------------------------------------------------------------------------

\begin{frame}{}
\end{frame}

%-------------------------------------------------------------------------------
\subsection{Gillespie's SSA in MABS: related work}
%-------------------------------------------------------------------------------


\begin{frame}{}
\end{frame}

\end{document}